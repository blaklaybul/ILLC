\documentclass[]{book}

%These tell TeX which packages to use.
\usepackage{array,epsfig}
\usepackage{amsmath}
\usepackage{amsfonts}
\usepackage{amssymb}
\usepackage{amsxtra}
\usepackage{amsthm}
\usepackage{mathrsfs}
\usepackage{color}

%Here I define some theorem styles and shortcut commands for symbols I use often
\theoremstyle{definition}
\newtheorem{defn}{Definition}
\newtheorem{thm}{Theorem}
\newtheorem{cor}{Corollary}
\newtheorem*{rmk}{Remark}
\newtheorem{lem}{Lemma}
\newtheorem*{joke}{Joke}
\newtheorem{ex}{Example}
\newtheorem*{soln}{Solution}
\newtheorem{prop}{Proposition}

%Pagination stuff.
\setlength{\topmargin}{-.3 in}
\setlength{\oddsidemargin}{0in}
\setlength{\evensidemargin}{0in}
\setlength{\textheight}{9.in}
\setlength{\textwidth}{6.5in}
\pagestyle{empty}



\begin{document}


\begin{center}
{\Large Statistical Learning Theory, Exercise 2}\\
Michael Hirsch\\ %name here
January 13, 2015 %date here
\end{center}

\vspace{0.2 cm}


\subsection*{Membership Processes}

Consider the membership process defined by the set of all intervals $[a, b] \subseteq \mathbb{R}$


\begin{enumerate}
\item\label{norms}

Compute $N(t)$ for $t = 1, 2, 3, 4$.

\begin{proof}[\unskip\nopunct]

Without loss of generality, we will assume that our questions are ordered. That is, $\forall x \geq 1, x_{t} > x_{t-1}$.

$$N(1) = 2$$
$$N(2) = 4$$
$$N(3) = 7$$
$$N(4) = 11$$



\end{proof}

\item

Find a general formula for $N(t)$.

\begin{proof}[\unskip\nopunct]

After some mathematical manipulation, the formula for $N(t)$ appears to be:

$$N(t) = \dfrac{t^{2}}{2} + \dfrac{t}{2} + 1 $$

\end{proof}

\item	

Compare $N(t)$ to $\Phi(2,t)$.

\begin{proof}[\unskip\nopunct]
		
$$\Phi(2,t) = {t \choose 0} + {t \choose 1} + {t \choose 2} = \dfrac{1}{2}(t^{2}+t+2) $$

This is equal to the $N(t)$ that was computed above.

\end{proof}

\end{enumerate}




\end{document}


