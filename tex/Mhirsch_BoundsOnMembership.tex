\documentclass[]{book}

%These tell TeX which packages to use.
\usepackage{array,epsfig}
\usepackage{amsmath}
\usepackage{amsfonts}
\usepackage{amssymb}
\usepackage{amsxtra}
\usepackage{amsthm}
\usepackage{mathrsfs}
\usepackage{color}

%Here I define some theorem styles and shortcut commands for symbols I use often
\theoremstyle{definition}
\newtheorem{defn}{Definition}
\newtheorem{thm}{Theorem}
\newtheorem{cor}{Corollary}
\newtheorem*{rmk}{Remark}
\newtheorem{lem}{Lemma}
\newtheorem*{joke}{Joke}
\newtheorem{ex}{Example}
\newtheorem*{soln}{Solution}
\newtheorem{prop}{Proposition}

%Pagination stuff.
\setlength{\topmargin}{-.3 in}
\setlength{\oddsidemargin}{0in}
\setlength{\evensidemargin}{0in}
\setlength{\textheight}{9.in}
\setlength{\textwidth}{6.5in}
\pagestyle{empty}



\begin{document}


\begin{center}
{\Large Statistical Learning Theory, Exercise 4}\\
Michael Hirsch\\ %name here
January 15, 2015 %date here
\end{center}

\vspace{0.2 cm}


\subsection*{Bounds on Membership Uncertainty}

A sample size of $2t = 2 \times 10^{4}$ is drawn from some distribution, and this sample is then randomly split up into two half-samples of size $t = 10^{4}$


\begin{enumerate}
\item\label{norms}

For any specific event $A$, these two half-samples define two frequencies, $f_{1}(A)$ and $f_{2}(A)$. Find an explicit upper bound on the probability that $|f_{1}(A) - f_{2}(A)| > 0.1$.

\begin{proof}[\unskip\nopunct]

In this scenario, we have two samples both of size $t = 10^{4}$, with the total population size, $T$, being unknown, as well as the number of successes in both the population and the sample. 

Further, we know that $E[s] = S/T \times 10^{4}$ in both samples.

Since we can assume that $S$ is fixed, we have the equality:

$$Pr\{|f_{1}(A) - f_{2}(A)| > 0.1\} = Pr\{|f_{1}(A) - E[f_{1}(A)]| > 0.05\} \leq 2 \times exp(-2t(\dfrac{\varepsilon}{2})^{2})$$

\end{proof}

\item

We now make such a comparison for each $\Phi(3,2 \times 10^{4}$ different sets. Find an explicit upper bound on the probability that $|f_{1}(A) - f_{2}(A)| > \varepsilon$ for at least one $A$.

\begin{proof}[\unskip\nopunct]

\end{proof}


\end{enumerate}


\end{document}


