\documentclass[]{book}

%These tell TeX which packages to use.
\usepackage{array,epsfig}
\usepackage{amsmath}
\usepackage{amsfonts}
\usepackage{amssymb}
\usepackage{amsxtra}
\usepackage{amsthm}
\usepackage{mathrsfs}
\usepackage{color}

%Here I define some theorem styles and shortcut commands for symbols I use often
\theoremstyle{definition}
\newtheorem{defn}{Definition}
\newtheorem{thm}{Theorem}
\newtheorem{cor}{Corollary}
\newtheorem*{rmk}{Remark}
\newtheorem{lem}{Lemma}
\newtheorem*{joke}{Joke}
\newtheorem{ex}{Example}
\newtheorem*{soln}{Solution}
\newtheorem{prop}{Proposition}

%Pagination stuff.
\setlength{\topmargin}{-.3 in}
\setlength{\oddsidemargin}{0in}
\setlength{\evensidemargin}{0in}
\setlength{\textheight}{9.in}
\setlength{\textwidth}{6.5in}
\pagestyle{empty}



\begin{document}


\begin{center}
{\Large Statistical Learning Theory, Exercise 4}\\
Michael Hirsch\\ %name here
January 15, 2015 %date here
\end{center}

\vspace{0.2 cm}


\subsection*{Bounds on Membership Uncertainty}

A sample size of $2t = 2 \times 10^{4}$ is drawn from some distribution, and this sample is then randomly split up into two half-samples of size $t = 10^{4}$


\begin{enumerate}
\item\label{norms}

For any specific event $A$, these two half-samples define two frequencies, $f_{1}(A)$ and $f_{2}(A)$. Find an explicit upper bound on the probability that $|f_{1}(A) - f_{2}(A)| > 0.1$.

\begin{proof}[\unskip\nopunct]

In this scenario, we have two samples both of size $t = 10^{4}$. Further, we know that $E[s_{1}] = E[s_{2}] = S/T \times 10^{4}$ in both samples, where S and T are unknown. Let us assume that $S$ and $T$ are fixed, we have the equality:
	\begin{equation*}
		\begin{split}
		Pr\{|f_{1}(A) - f_{2}(A)| > 0.1\} & = Pr\{|f_{1}(A) - E[f_{1}(A)]| > 0.05\} \\
		& = Pr\{|s_{1} - St/T| > 0.05\} \\
		& = Pr\{|s_{1}/t - S/T| > 0.05/t\} \\
		& = Pr\{|s_{1}/t - E[s_{1}/t]| > 0.05/t\} \\
		& < 2e^{-2(0.05)^{2}10^{4}} = 2e^{-50} \approx 1.93 \times 10^{-22}
		\end{split}
	\end{equation*}

That is, there is a very small probability that the frequencies will differ by more than 10\% of the mean
\end{proof}

\item

We now make such a comparison for each $\Phi(3,2 \times 10^{4})$ different sets. Find an explicit upper bound on the probability that $|f_{1}(A) - f_{2}(A)| > \varepsilon$ for at least one $A$.

\begin{proof}[\unskip\nopunct]
The union bound, also known as Boole's Inequality, is formulated as follows, where $i \in [1,\Phi(3,2\times10^{4})]$:

For any countable number of countable events $A_{1}, A_{2}, A_{3}, \ldots$ we have:
	\begin{equation*}
		\begin{split}
			P(\bigcup_{i} A_{i}) \} \leq \sum_{i}P(A_{i}) & = {\Phi(3,2\times10^{4}) \choose 2} \times 2e^{-50} \\
			& = \Bigg( {{1000 \choose 0} + {1000 \choose 1} + {1000 \choose 2} + {1000 \choose 3} \choose 2}\Bigg) \times e^{-50} \\
			& = {1+1000+49950+166167000 \choose 2}\times e^{-50} \\
			& \approx 1.3322 \times 10^{-6}
		\end{split}
	\end{equation*} 
	
Here, we have taken $\varepsilon = 0.1$ as in the first example. Each event $A_{i}$ will actually be the event in which we compare the frequencies defined by two samples. Because of this, there will actually be ${\Phi(3,2\times10^{4}) \choose 2}$ such events. Since each comparison will happen twice, we will need to divide this factor by 2.
\end{proof}


\end{enumerate}


\end{document}


