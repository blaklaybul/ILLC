\documentclass[]{book}

%These tell TeX which packages to use.
\usepackage{array,epsfig}
\usepackage{amsmath}
\usepackage{amsfonts}
\usepackage{amssymb}
\usepackage{amsxtra}
\usepackage{amsthm}
\usepackage{mathrsfs}
\usepackage{color}

%Here I define some theorem styles and shortcut commands for symbols I use often
\theoremstyle{definition}
\newtheorem{defn}{Definition}
\newtheorem{thm}{Theorem}
\newtheorem{cor}{Corollary}
\newtheorem*{rmk}{Remark}
\newtheorem{lem}{Lemma}
\newtheorem*{joke}{Joke}
\newtheorem{ex}{Example}
\newtheorem*{soln}{Solution}
\newtheorem{prop}{Proposition}

%Pagination stuff.
\setlength{\topmargin}{-.3 in}
\setlength{\oddsidemargin}{0in}
\setlength{\evensidemargin}{0in}
\setlength{\textheight}{9.in}
\setlength{\textwidth}{6.5in}
\pagestyle{empty}



\begin{document}


\begin{center}
{\Large Statistical Learning Theory, Exercise 1}\\
Michael Hirsch\\ %You should put your name here
January 12, 2015 %You should write the date here.
\end{center}

\vspace{0.2 cm}


\subsection*{Coin Flipping Inference}

\begin{enumerate}
\item\label{norms}
How many times should we flip the coin in order to achieve a precision of $\epsilon = 10^{-2}$ and error probability of $\alpha = 1/20$?
\begin{proof}
	We will solve for $t$ in the equation $1/4t\epsilon^{2} = \alpha$ by plugging in the values for $\epsilon$ and $\alpha$.

	$$ \dfrac{1}{4t10^{-4}} = \dfrac{1}{20}$$
	$$ \dfrac{2500}{t} = \dfrac{1}{20}$$
	$$ t = 500000$$
\end{proof}

\item	If we flip the coint $ t = 10^{3}$ times and want the error probability to be less than $\alpha = 1/20$, what precision level can we obtain?
\begin{proof} Again, we solve the same equation using different values, this time noting that our epsilon value will be a bound on the precision.
	$$ \dfrac{1}{4000\epsilon^{2}} = \dfrac{1}{20}$$
	$$ 4000\epsilon^{2} = 20 $$
	$$ \epsilon ^{2} = \dfrac{20}{4000}$$
	$$ \epsilon ^{2} = \dfrac{1}{200}$$
	$$ \epsilon  = \sqrt{\dfrac{1}{200}}$$
	$$ \epsilon  = \dfrac{1}{\sqrt{200}}$$
	$$ \epsilon  = \dfrac{1}{10\sqrt{2}}$$
\end{proof}

\item	If we flip the coin $t = 10^{3}$ times, what is the probability that the empirical frequency of heads deviates from the probability by more that $\epsilon = 10 ^ {-2}$?
\begin{proof}
		$$ Pr\{| \dfrac{S_{t}}{t} - \mu | > 1/100 \} \leq \dfrac{0.25/1000}{10^{-4}} = 2.5$$
		
\end{proof}

It appears that in this example, the Chebyshev bound is not effective given the number of trials and our desired degree of precision. A probaility of at most 2.5 indicates that we are entirely in the dark when predicting the empirical frequency of the number of heads.

\end{enumerate}




\end{document}


